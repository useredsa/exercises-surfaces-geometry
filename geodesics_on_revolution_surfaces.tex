%%%%%%%%%%%%%%%%%%%%%%%%%%%%%%%%%%%%%%%%%
% fphw Assignment
% LaTeX Template
% Version 1.0 (27/04/2019)
%
% This template originates from:
% https://www.LaTeXTemplates.com
%
% Authors:
% Class by Felipe Portales-Oliva (f.portales.oliva@gmail.com) with template 
% content and modifications by Vel (vel@LaTeXTemplates.com)
%
% Template (this file) License:
% CC BY-NC-SA 3.0 (http://creativecommons.org/licenses/by-nc-sa/3.0/)
%
%%%%%%%%%%%%%%%%%%%%%%%%%%%%%%%%%%%%%%%%%

%----------------------------------------------------------------------------------------
%    PACKAGES AND OTHER DOCUMENT CONFIGURATIONS
%----------------------------------------------------------------------------------------

\documentclass[
    12pt, % Default font size, values between 10pt-12pt are allowed
    %letterpaper, % Uncomment for US letter paper size
    %spanish, % Uncomment for Spanish
]{fphw}

% Template-specific packages
\usepackage[utf8]{inputenc} % Required for inputting international characters
\usepackage[T1]{fontenc} % Output font encoding for international characters
\usepackage{fontspec,unicode-math} % Required for using utf8 characters in math mode
\usepackage{parskip}  % To add extra space between paragraphs
% \usepackage{mathpazo} % Use the Palatino font
\usepackage{graphicx} % Required for including images
\usepackage{booktabs} % Better horizontal rules in tables
\usepackage{hyperref} % For links (both internal and external)
% \usepackage{listings} % Required for insertion of code
\usepackage{enumerate}% To modify the enumerate environment
\usepackage{cleveref} % Better \ref command -> \cref
\usepackage{import}   % This 4 packages and the command allow importing pdf
\usepackage{xifthen}  % figures generated with inkscape
\usepackage{pdfpages} % Source: https://castel.dev/post/lecture-notes-2/
\usepackage{mathtools}
\usepackage{wrapfig}
\usepackage{cancel}
\usepackage{transparent}
\newcommand{\incfig}[1]{%
    \def\svgwidth{0.95\columnwidth}
    \small
        \import{./images/}{#1.pdf_tex}
}

\setlength{\parindent}{15pt}
\setlength{\headheight}{22.66pt}

%----------------------------------------------------------------------------------------
%    ASSIGNMENT INFORMATION
%----------------------------------------------------------------------------------------

\title{Task 2 \\ Geodesics on Revolution Surfaces} % Assignment title

\author{Emilio Domínguez Sánchez} % Student name

\date{March 7th, 2021} % Due date

\institute{University of Murcia \\ Faculty of Mathematics} % Institute or school name

\class{Geometría Global de Superficies} % Course or class name

\professor{Dr. Luis J. Alías Linares} % Professor or teacher in charge of the assignment

%----------------------------------------------------------------------------------------
%    Definitions
%----------------------------------------------------------------------------------------

\usepackage{physics}
\newcommand{\R}{\mathbb{R}}
\newcommand{\inner}[2]{\left\langle #1, \; #2 \right\rangle}
\newcommand{\vfield}{\mathfrak{X}}

\DeclareDocumentCommand\covariantderivative{ s o m g d() }
{ % Covariant derivative
	% s: star for \flatfrac flat covariant derivative
	% o: optional n for nth covariant derivative
	% m: mandatory (x in Df/dx)
	% g: optional (f in Df/dx)
	% d: long-form D/dx(...)
	\IfBooleanTF{#1}
	{\let\fractype\flatfrac}
	{\let\fractype\frac}
	\IfNoValueTF{#4}
	{
		\IfNoValueTF{#5}
		{\fractype{D \IfNoValueTF{#2}{}{^{#2}}}{\diffd #3\IfNoValueTF{#2}{}{^{#2}}}}
		{\fractype{D \IfNoValueTF{#2}{}{^{#2}}}{\diffd #3\IfNoValueTF{#2}{}{^{#2}}} \argopen(#5\argclose)}
	}
	{\fractype{D \IfNoValueTF{#2}{}{^{#2}} #3}{\diffd #4\IfNoValueTF{#2}{}{^{#2}}}}
}
\DeclareDocumentCommand\cdv{}{\covariantderivative} % Shorthand for \covariantderivative

\begin{document}

\maketitle % Output the assignment title, created automatically using the information in the custom commands above

%----------------------------------------------------------------------------------------
%    ASSIGNMENT CONTENT
%----------------------------------------------------------------------------------------

\section*{Problem}

\begin{problem}
    Consider the revolution surface $S$ generated by rotating
    the regular curve $\mqty(f(u) & 0 & g(u))$
    around the axis $\operatorname{OZ}$,
    given by the parametrization
    \begin{equation*}
        X(u, v) = \mqty(f(u)\cos v & f(u)\sin v & g(u)),
    \end{equation*}
    where $f(u) > 0$ and $f'(u)^2 + g'(u)^2 > 0$.

    Let $α(s) = X(u(s), v(s))$ be a geodesic parametrized by arc
    over a certain interval $I \subset \R$, $α : I \to S$.
    Prove that the function $f(u(s))\sin φ(s)$ is constant,
    where $φ(s)$ is the angle described by the vectors $X_u(u, v)$ and $α'$.
\end{problem}

%----------------------------------------------------------------------------------------

\subsection*{Answer}

    We will first determine an expression for $\sin φ$ that lets us
determine what the condition of the statement is equivalent to.
\begin{equation*}
    \sin φ = \frac{X_u \wedge α'}{\norm{X_u}\norm{α'}},
\end{equation*}
where $\wedge : T_αS \times T_αS \to \R$ denotes
the vector product in $T_αS$ and
the derivative of $α$ is $α' = X_uu' + X_vv'$.
We will now compute the values specific to this surface.
\begin{align*}
    X_u(u, v) & {} = \mqty(f'(u)\cos v & f'(u)\sin v & g'(u)), \\
    X_v(u, v) & {} = \mqty(-f(u)\sin v & f(u)\cos v & 0), \qq{and} \\
    \inner{X_u}{X_v} & {} = 0.
\end{align*}
Hence\footnote{
    The vector product is the same as the determinant,
    its sign depends on a choice of basis
    and it satistifies
    $\inner{v}{w}^2 + \qty(v \wedge w)^2 = \norm{v}^2\norm{w}^2$.
    It is natural to orientate $S$ via $X$,
    in which case $X_u \wedge X_v$ is positive and
    $X_v \wedge X_u$ is negative.
}, $X_u \wedge X_v = \norm{X_u}\norm{X_v}$ and
\begin{multline*}
    \sin φ =
    \frac{X_u \wedge α'}{\norm{X_u}\norm{α'}} =
    \frac{X_u(u, v) \wedge \qty(X_u(u, v)u' + X_v(u, v)v')}
        {\norm{X_u(u, v)}} =
    % \frac{\norm{X_u(u, v)}\norm{X_v(u, v)}v'}{\norm{X_u(u, v)}} = \\
    \norm{X_v(u, v)}v' =
    f(u)v'.
\end{multline*}
It follows that $f(u)\sin φ = f(u)^2v'$ is constant if and only if
\begin{equation*}
    2f(u)f'(u)u'v' + f(u)^2v'' = 0.
\end{equation*}

    We now need to make use of the fact that $α$ is a geodesic
to prove that the differential equation holds.
We need some property unique to geodesics.
Oddly enough, the ODE is very similar to the differential equation
\begin{equation*}
    Γ_{11}^2u'u' + 2Γ_{12}^2u'v' + Γ_{22}^2v'v' + v'' = 0
\end{equation*}
a geodesic must satisfy.
Indeed, because we have
\begin{align*}
    X_v(u, v) & {} = \mqty(-f(u)\sin v & f(u)\cos v & 0), \\
    X_{uu}(u, v) & {} = \mqty(f''(u)\cos v & f''(u)\sin v & g''(u)), \\
    X_{uv}(u, v) & {} = \mqty(-f'(u)\sin v & f'(u)\cos v & 0) \qq{and} \\
    X_{vv}(u, v) & {} = \mqty(-f(u)\cos v & -f(u)\sin v & 0),
\end{align*}
the coefficients are
$Γ_{11}^2 = \frac{\inner{X_{uu}}{X_v}}{\norm{X_v}^2} = 0$,
$Γ_{12}^2(u, v) = \frac{f'(u)}{f(u)}$ and
$Γ_{22}^2 = 0$.
And now it is clear that both equations are equivalent,
\begin{equation*}
    2f(u)f'(u)u'v' + f(u)^2v'' = 0 \qq{and}
    \begin{gathered}
        Γ_{11}^2u'u' + 2Γ_{12}^2u'v' + Γ_{22}^2v'v' + v'' = \\
        2\frac{f'(u)}{f(u)}u'v' + v'' = 0,
    \end{gathered}
\end{equation*}
provided that $f(u) \ne 0$.

%----------------------------------------------------------------------------------------

\end{document}
