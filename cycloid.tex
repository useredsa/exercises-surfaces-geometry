%%%%%%%%%%%%%%%%%%%%%%%%%%%%%%%%%%%%%%%%%
% fphw Assignment
% LaTeX Template
% Version 1.0 (27/04/2019)
%
% This template originates from:
% https://www.LaTeXTemplates.com
%
% Authors:
% Class by Felipe Portales-Oliva (f.portales.oliva@gmail.com) with template 
% content and modifications by Vel (vel@LaTeXTemplates.com)
%
% Template (this file) License:
% CC BY-NC-SA 3.0 (http://creativecommons.org/licenses/by-nc-sa/3.0/)
%
%%%%%%%%%%%%%%%%%%%%%%%%%%%%%%%%%%%%%%%%%

%----------------------------------------------------------------------------------------
%    PACKAGES AND OTHER DOCUMENT CONFIGURATIONS
%----------------------------------------------------------------------------------------

\documentclass[
    12pt, % Default font size, values between 10pt-12pt are allowed
    %letterpaper, % Uncomment for US letter paper size
    %spanish, % Uncomment for Spanish
]{fphw}

% Template-specific packages
\usepackage[utf8]{inputenc} % Required for inputting international characters
\usepackage[T1]{fontenc} % Output font encoding for international characters
\usepackage{fontspec,unicode-math} % Required for using utf8 characters in math mode
\usepackage{parskip}  % To add extra space between paragraphs
\usepackage{mathpazo} % Use the Palatino font
\usepackage{graphicx} % Required for including images
\usepackage{booktabs} % Better horizontal rules in tables
\usepackage{hyperref} % For links (both internal and external)
% \usepackage{listings} % Required for insertion of code
\usepackage{enumerate}% To modify the enumerate environment
\usepackage{cleveref} % Better \ref command -> \cref
\usepackage{import}   % This 4 packages and the command allow importing pdf
\usepackage{xifthen}  % figures generated with inkscape
\usepackage{pdfpages} % Source: https://castel.dev/post/lecture-notes-2/
\usepackage{mathtools}
\usepackage{transparent}
\newcommand{\incfig}[1]{%
    \def\svgwidth{0.95\columnwidth}
    \small
        \import{./images/}{#1.pdf_tex}
}

\setlength{\parindent}{15pt}

%----------------------------------------------------------------------------------------
%    ASSIGNMENT INFORMATION
%----------------------------------------------------------------------------------------

\title{Task 2 \\ The Cycloid} % Assignment title

\author{Emilio Domínguez Sánchez} % Student name

\date{October 14th, 2020} % Due date

\institute{University of Murcia \\ Faculty of Mathematics} % Institute or school name

\class{Geometría de Superficies} % Course or class name

\professor{Dr. Pascual Lucas Saorin} % Professor or teacher in charge of the assignment

%----------------------------------------------------------------------------------------
%    Definitions
%----------------------------------------------------------------------------------------

\usepackage{physics}
\DeclareMathOperator{\len}{len}
\newcommand{\R}{\mathbb{R}}
\newcommand{\N}{\mathbb{N}}

\begin{document}

\maketitle % Output the assignment title, created automatically using the information in the custom commands above

%----------------------------------------------------------------------------------------
%    ASSIGNMENT CONTENT
%----------------------------------------------------------------------------------------

\section*{Problem}

\begin{problem}
    Consider a circunference of radius $1$ that rolls along the $x$ axis.
The curve described by a point of the circunference as it rolls is called cycloid,
one of the most important classical curves.

\begin{center}
    \incfig{cycloid}
\end{center}

    Find the parametrization $α : \R \to \R^2$ of the cycloid,
its singular points,
and the length for a complete turn of the circunference.
\end{problem}

%----------------------------------------------------------------------------------------

\subsection*{Answer}

    It seems natural to parametrize the cycloid in terms of
the angle rotation $θ$ of the circunference.
Because then,
the distance travelled by the center of the circunference,
which moves horizontally at height $1$,
and the angle of the selected point are easy to parametrize
(see \cref{fig-distance-by-angle}).

\begin{figure}[h]
    \incfig{cycloid_circunference_rolling}
    \caption{Distance travelled by a rolling circunference
    with respect to the rotated angle.}
    \label{fig-distance-by-angle}
\end{figure}

\noindent
The distance travelled by the center has to be equal to
the length of the arc of circunference that has been in contact with the «ground»
(the $x$ axis).
That is, the radius times the angle of turn,
which is equal to the radius of turn in our problem.
Hence, the parametrization for the position of the center of the circunference is

\begin{equation*}
    c(θ) = \qty(θ, 1).
\end{equation*}

\noindent
It is true this reasoning depends on the physical interpretation of the movement,
but this cannot be avoided as the problem's statement does not describe
what we understand as rolling through the $x$ axis with a formula.

\noindent
For $θ = 0$, the selected point forms an angle of $\frac{3π}{2}$
and rotates in counterclockwise sense,
which makes its position

\begin{multline*}
% The use of spacing here goes aggainst the separation of content and formatting
% but the output didn't look right otherwise.
    α(θ) = \\
    \begin{split}
        c(θ) + \qty\Big(\cos(\frac{3π}{2}-θ), \sin(\frac{3π}{2}-θ)) =
        & \qty\Big(\text{using } \cos(ρ - \frac{π}{2}) = \ \; \sin ρ) = \\
        c(θ) + \qty\Big(\sin(2π-θ), \cos(\frac{π}{2}-θ)) =
        & \qty\Big(\text{using } \sin\Big(-ρ\Big) \ \,  = -\sin ρ) = \\
        c(θ) + \qty\Big(-\sin θ, \cos(\frac{π}{2}-θ)) =
        & \qty\Big(\text{using } \cos(\frac{π}{2}-ρ) = -\cos ρ) = \\
        c(θ) + \qty\Big(-\sin θ, -\cos(θ)) =
        & \qty\Big(θ, 1) + \qty\Big(-\sin θ, -\cos θ) = \\
    \end{split} \\
    \qty\Big(θ - \sin θ, 1 - \cos θ).\footnotemark
\end{multline*}

\footnotetext{
    A nice list of trigonometric identities can be found at
\href{https://en.wikipedia.org/wiki/List_of_trigonometric_identities}{wikipedia.org}.
}

    To find the singular points and the length of the cycloid,
we will need to compute the derivative of $α$.

\begin{align*}
    \dv{α}{θ} &= \qty\Big(1 - \cos θ, \sin θ). \\
    \norm{\dv{α}{θ}} &= \sqrt{\qty(1 - \cos θ)^2 + \qty(\sin θ)^2} = \sqrt{2 - 2\cos θ}.
\end{align*}

    We can now solve the equation $\dv{α}{θ} = 0$ to find the singular points.

\begin{gather*}
    \dv{α}{θ} = 0 \\
    \qty\Big(1 - \cos θ, \sin θ) = 0 \\
    1 - \cos θ = 0 \iff θ = 0 + 2πk \quad (k \in \N) \implies \sin θ = 0,
\end{gather*}

\noindent
concluding that the singular points are $\{2πk\}_{k \in \N}$.

    And finally, to find the length for a complete turn of the circunference,
we need to integrate an interval of size $2π$
$\qty(\int_{θ_0}^{θ_0+2π} \norm{\dv{α}{ϴ}} \dd{θ})$.%\footnotemark.
We will choose to integrate the interval $[0, 2π]$.

% \footnotetext{ Removed: The curve has unregular points but they aren't singular
%     Even though the curve contains singular points,
% the length can still be defined for differentiable curves in almost every point
% in just as with differentiable functions and calculated by integrating the derivative.
% However, the choice of $θ_0 = 0$ makes the tooling unnecesary for this problem.
% } %TODO check needed: subject FVVIII

\begin{multline*}
    \int_0^{2π} \norm{\dv{α}{ϴ}} \dd{θ} =
    \int_0^{2π} \sqrt{2-2\cos θ} \dd{θ} = \\
    \begin{multlined}
        \int_0^{2π} \sqrt{2-2\qty(\cos^2 \frac{θ}{2} - \sin^2 \frac{θ}{2})} \dd{θ} =
        \int_0^{2π} \sqrt{2-2\qty(1 - 2\sin^2 \frac{θ}{2})} \dd{θ} = \\
        \int_0^{2π} \sqrt{4\sin^2 \frac{θ}{2}} \dd{θ} =
        \Big(\text{When } \frac{θ}{2} \in [0, π], \:
                    \abs{\sin \frac{θ}{2}} = \sin \frac{θ}{2}\Big) =
        \int_0^{2π} 2\sin\frac{θ}{2} \dd{θ} = \\
        4\int_0^{2π} \frac{1}{2}\sin\frac{θ}{2} \dd{θ} =
        4\eval[-\cos\frac{θ}{2}|_0^{2π} =
    \end{multlined} \\
    8.
\end{multline*}

% \newpage
% \section*{Integrating without the half angle identity}



%----------------------------------------------------------------------------------------

\end{document}
