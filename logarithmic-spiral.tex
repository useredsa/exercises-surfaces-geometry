%%%%%%%%%%%%%%%%%%%%%%%%%%%%%%%%%%%%%%%%%
% fphw Assignment
% LaTeX Template
% Version 1.0 (27/04/2019)
%
% This template originates from:
% https://www.LaTeXTemplates.com
%
% Authors:
% Class by Felipe Portales-Oliva (f.portales.oliva@gmail.com) with template 
% content and modifications by Vel (vel@LaTeXTemplates.com)
%
% Template (this file) License:
% CC BY-NC-SA 3.0 (http://creativecommons.org/licenses/by-nc-sa/3.0/)
%
%%%%%%%%%%%%%%%%%%%%%%%%%%%%%%%%%%%%%%%%%

%----------------------------------------------------------------------------------------
%	PACKAGES AND OTHER DOCUMENT CONFIGURATIONS
%----------------------------------------------------------------------------------------

\documentclass[
	12pt, % Default font size, values between 10pt-12pt are allowed
	%letterpaper, % Uncomment for US letter paper size
	%spanish, % Uncomment for Spanish
]{fphw}

% Template-specific packages
\usepackage[utf8]{inputenc} % Required for inputting international characters
\usepackage[T1]{fontenc} % Output font encoding for international characters
\usepackage{fontspec,unicode-math} % Required for using utf8 characters in math mode
\usepackage{parskip}  % To add extra space between paragraphs
\usepackage{mathpazo} % Use the Palatino font
\usepackage{graphicx} % Required for including images
\usepackage{booktabs} % Required for better horizontal rules in tables
% \usepackage{listings} % Required for insertion of code
\usepackage{enumerate}% To modify the enumerate environment
\setlength{\parindent}{15pt}

%----------------------------------------------------------------------------------------
%	ASSIGNMENT INFORMATION
%----------------------------------------------------------------------------------------

\title{Task 1 \\ The Logarithmic Spiral} % Assignment title

\author{Emilio Domínguez Sánchez} % Student name

\date{October 3rd, 2020} % Due date

\institute{University of Murcia \\ Faculty of Mathematics} % Institute or school name

\class{Geometría de Superficies} % Course or class name

\professor{Dr. Pascual Lucas Saorin} % Professor or teacher in charge of the assignment

%----------------------------------------------------------------------------------------
%	Definitions
%----------------------------------------------------------------------------------------

\usepackage{physics}
\DeclareMathOperator{\len}{len}
\newcommand{\R}{\mathbb{R}}

\begin{document}

\maketitle % Output the assignment title, created automatically using the information in the custom commands above

%----------------------------------------------------------------------------------------
%	ASSIGNMENT CONTENT
%----------------------------------------------------------------------------------------

\section*{Problem}

\begin{problem}
    Let $α : \R → \R^2$ be the parametrized curve given by $α(t) = ae^{bt}(\cos t, \sin t)$,
where $a > 0$ and $b < 1$.
Find the arc length function and a reparametrization with respect to arc length.
\end{problem}

%----------------------------------------------------------------------------------------

\subsection*{Answer}

    The arc length function can be as the integral of the norm of the derivative.
That is,

\begin{multline*}
    \len(t_0, τ) =
    \int_{t_0}^τ \norm{\dv{α}{t}} \dd{t} = \\
    \int_{t_0}^τ \norm{ae^{bt} \qty(b\cos t - \sin t, b\sin t + \cos t)} \dd{t} =
    \int_{t_0}^τ ae^{bt}\sqrt{b^2 + 1} \dd{t} = \\
    \left\{ \begin{split}
        (b = 0) & &
        = a\eval[τ|_{t_0}^τ
        = a(τ-t_0). \\
        (b ≠ 0) & &
        = \frac{a\sqrt{b^2 + 1}}{b} \eval[e^{bt}|_{t_0}^τ =
        k (e^{bτ} - e^{bt_0}). \\
    \end{split} \right.
\end{multline*}

\noindent
where we have defined $k$ as $\frac{a\sqrt{b^2 +1}}{b}$ for convenience.

    When $b = 0$, the spiral becomes a circunference
that is traversed at the constant speed $a$.
This degenerate case can be reparametrized as

\begin{align*}
    \vectorunit*{α}(t) & =
    a\qty(\cos \frac{t}{a}, \sin \frac{t}{a}), \\
    \norm{\dv{\vectorunit*{α}(t)}{t}} & =
    \norm{\qty(-\sin \frac{t}{a}, \cos \frac{t}{a})} = 1.
\end{align*}

    For the case $b \neq 0$, owing to the fact that
the (timed by $t_0$) position can be chosen arbitrarily,
and, because either
$\lim_{t_0 \to -∞} e^{bt_0} = 0$ if $b > 0$ or
$\lim_{t_0 \to ∞} e^{bt_0} = 0$ if $b < 0$,
we can consider the integral when $t_0 = -∞$ or $t_0 = ∞$.
Hence, and for simplicity, we will assume $b > 0$ from here onwards
and write the length as $\len(-∞, τ) = \len(τ) = ke^{bτ}$.

    To reparametrizate the curve with respect to its length,
we guess that we need to find the function

\begin{align*}
    \R & \to \R^2 \\
    t & \mapsto α(p_t | \len(p_t) = t), \\
\end{align*}

which ought to be

\begin{align*}
    \R & \to \R^2 \\
    t & \mapsto α(\len^{-1}(p_t)). \\
\end{align*}

\noindent
We have seen in class a theorem that asserts that the inverse and this reparametrization
always exist when $\dv{α}{t} ≠ 0$.
For our particular problem, it is enough to clear $τ$ from the relation $len(τ) = ke^{bτ}$,
giving
%TODO clear = despejar?

\begin{align*}
    \len(τ) & = ke^{bτ} \\
    τ & = \frac{\ln \len(τ) - \ln k}{b}\\
    \len^{-1}(τ) & = \frac{\ln τ - \ln k}{b}.
\end{align*}

\noindent
Thus, the formula for the reparametrization would be

\begin{multline*}
    \vectorunit*{α}(t) =
    α\qty(\len^{-1}(t)) =
    α\qty(\frac{\ln t - \ln k}{b}) = \\
    ae^{b\frac{\ln t - \ln k}{b}} \qty(\cos \frac{\ln t - \ln k}{b}, \sin \frac{\ln t - \ln k}{b}) =
    \qty( k = \frac{a\sqrt{b^2 +1}}{b} ) = \\
    \frac{bt}{\sqrt{b^2+1}} \qty(\cos \frac{\ln t - \ln k}{b}, \sin \frac{\ln t - \ln k}{b}).
\end{multline*}

%----------------------------------------------------------------------------------------

\subsection*{Extra}

    As you may have noticed, the expression for $\vectorunit*{α}$
(which is in polar form)
shows that the radius of the spiral does not depend on the choice of $a$.
And although the angle does (via $k$),
the derivative of the angle doesn't.
What this means is that the set of curves with the same $b$ are all similar by rotation.
We can check that a rotation of a logarithmic spiral does, indeed,
give another logarithmic spiral with the same $b$ parameter.
That is, a scaled version of the original curve.
Similarly, a scaled version of the original curve can be obtained via a rotation.
This property of the logarithmic spiral is called self-similarity.

%----------------------------------------------------------------------------------------

\end{document}
