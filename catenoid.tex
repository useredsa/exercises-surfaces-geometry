%%%%%%%%%%%%%%%%%%%%%%%%%%%%%%%%%%%%%%%%%
% fphw Assignment
% LaTeX Template
% Version 1.0 (27/04/2019)
%
% This template originates from:
% https://www.LaTeXTemplates.com
%
% Authors:
% Class by Felipe Portales-Oliva (f.portales.oliva@gmail.com) with template 
% content and modifications by Vel (vel@LaTeXTemplates.com)
%
% Template (this file) License:
% CC BY-NC-SA 3.0 (http://creativecommons.org/licenses/by-nc-sa/3.0/)
%
%%%%%%%%%%%%%%%%%%%%%%%%%%%%%%%%%%%%%%%%%

%----------------------------------------------------------------------------------------
%	PACKAGES AND OTHER DOCUMENT CONFIGURATIONS
%----------------------------------------------------------------------------------------

\documentclass[
	12pt, % Default font size, values between 10pt-12pt are allowed
	%letterpaper, % Uncomment for US letter paper size
	%spanish, % Uncomment for Spanish
]{fphw}

% Template-specific packages
\usepackage[utf8]{inputenc} % Required for inputting international characters
\usepackage[T1]{fontenc} % Output font encoding for international characters
\usepackage{fontspec,unicode-math} % Required for using utf8 characters in math mode
\usepackage{parskip}  % To add extra space between paragraphs
\usepackage{graphicx} % Required for including images
\usepackage{booktabs} % Required for better horizontal rules in tables
% \usepackage{listings} % Required for insertion of code
\usepackage{enumerate}% To modify the enumerate environment
\usepackage{import}   % This 4 packages and the command allow importing pdf
\usepackage{xifthen}  % figures generated with inkscape
\usepackage{pdfpages} % Source: https://castel.dev/post/lecture-notes-2/
\usepackage{physics}
\usepackage{transparent}
\newcommand{\incfig}[1]{%
    \def\svgwidth{0.45\columnwidth}
    \small
        \import{./images/}{#1.pdf_tex}
}
\setlength{\parindent}{15pt}
\setlength{\headheight}{22.66pt}

%----------------------------------------------------------------------------------------
%	ASSIGNMENT INFORMATION
%----------------------------------------------------------------------------------------

\title{Task 6 \\ A Catenoid} % Assignment title

\author{Emilio Domínguez Sánchez} % Student name

\date{May 23rd, 2021} % Due date

\institute{University of Murcia \\ Faculty of Mathematics} % Institute or school name

\class{Geometría Global de Superficies} % Course or class name

\professor{Dr. Luis J. Alías Linares} % Professor or teacher in charge of the assignment

%----------------------------------------------------------------------------------------
%	Definitions
%----------------------------------------------------------------------------------------

\usepackage{cleveref}
\newcommand{\R}{\mathbb{R}}
\newcommand{\inner}[2]{\left\langle #1, \; #2 \right\rangle}
\newcommand{\D}{\hat{S}}
\DeclareMathOperator{\area}{area}
\DeclareMathOperator{\sgn}{sgn}
\newcommand{\axis}[1]{\mathrm{#1}}

\begin{document}

\maketitle % Output the assignment title, created automatically using the information in the custom commands above

%----------------------------------------------------------------------------------------
%	ASSIGNMENT CONTENT
%----------------------------------------------------------------------------------------

\section*{Problem}

\begin{problem}
    Let $S \subset \R^3$ be the catenoid obtained by rotating the curve
    $α(t) = \mqty(\cosh{t} & 0 & t)$, $α : \R \to \R^3$,
    along the axis $\axis{OZ}$,
    that can be parametrized via
    \begin{equation*}
        X(θ, t) = \mqty(\cosh{t}\cos{θ} & \cosh{t}\sin{θ} & t),
    \end{equation*}
    $X : I \times R \subset \R^2 \to S$ (for an appropiate interval $I$
    on which $X$ is inyective).

    \begin{enumerate}
        \item \label{itm:minimal-umbilical} Prove that $S$ is a minimal surface.
        ¿Does it have any umbilical points?
        \item \label{itm:area} For every $a > 0$,
        find the area of $S_a = \qty{\mqty(x & y & z) \in S : \abs{z} < a}$.
        Notice that $S_a$ is the part of the catenoid contained between
        the two circumferences obtained by
        intersecting the catenoid with the planes $z = a$ and $z = -a$.
        \item  There is another (disconnected) minimal surface
        with the same surface as $S_a$:
        the one given by the two circles $D_+(a)$ and $D_-(a)$
        determined, respectively, by the circunferences $z = a$ and $z = -a$.
        Let $\D_a := D_+(a) \cup D_-(a)$.
        Find the area of $\D_a$ and prove that there exists a constant $L$ such that
        $\area(\D_a) < \area(S_a)$ if and only if $a > L$. 
    \end{enumerate}
\end{problem}

%----------------------------------------------------------------------------------------

We start by calculating the partial derivatives of $X$,
\begin{align*}
    X_θ(θ, t) &= \mqty(-\cosh{t}\sin{θ} & \cosh{t}\cos{θ} & 0) \qq{and} \\
    % X_t(θ, t) &= \mqty(\sinh{t}\cos{θ} & \sinh{t}\sin{θ} & 1),
    X_t(θ, t) &= \mqty(\;\sinh{t}\cos{θ}\hspace{0.75em} & \sinh{t}\sin{θ} & \,1),
\end{align*}
which we use to calculate $X_θ \cross X_t$ and its norm.
\begin{align*}
    X_θ(θ, t) \cross X_t(θ, t)\:\,
        &= \mqty(\cosh{t}\cos{θ} & \cosh{t}\sin{θ} & -\cosh{t}\sinh{t}) \\
    \norm{X_θ(θ, t) \cross X_t(θ, t)}
        &= \sqrt{\cosh^2 t + \cosh^2 t \sinh^2 t}
        % = \sqrt{\cosh^2 t(1 + \sinh^2 t)}
        = \sqrt{\cosh^4 t}
        = \cosh^2 t
\end{align*}
We will use the norm to calculate the area of $S_a$ and, later, the mean curvature.
\begin{align*}
    \area(S_a)
    &= \int_0^{2π}\int_{-a}^a \norm{X_θ(θ, t) \cross X_t(θ, t)} \dd{t}\dd{θ}
    = \int_0^{2π}\int_{-a}^a \cosh^2 t \dd{t}\dd{θ} \\
    &= 2π\int_{-a}^a \cosh^2 t \dd{t}
    = 2π\int_{-a}^a \frac{\cosh(2t) + 1}{2} \dd{t}
    = 2π\qty(\eval{\frac{\sinh(2t)}{4}}_{-a}^a + a) \\
    &= 2π\qty(\frac{\sinh(2a)}{2} + a).
\end{align*}
This completes \cref{itm:area}.

\newpage
%----------------------------------------------------------------------------------------

A caracterization of minimal surfaces is those whose mean curvature is $0$.
Expressing the normal $N(x, y, z)$ in terms of the coordinates of $S$ in $\R^3$,
\begin{equation*}
    \mqty(x(θ, t) & y(θ, t) & z(θ, t)) = X(θ, t),
\end{equation*}
the mean curvature can be calculated as $2H(x, y, z) = \divergence{N}(x, y, z)$.
We perform the desired change of domain,
\begin{align*}
    X(θ, t) &= \mqty(\cosh{t}\cos{θ} & \cosh{t}\sin{θ} & t)
        \sim \mqty(x & y & z), \\
    N(θ, t) &= \frac{1}{\cosh^2 t}
        \mqty(\cosh{t}\cos{θ} & \cosh{t}\sin{θ} & -\cosh{t}\sinh{t}), \\
    N(x, y, z) &= \frac{1}{x^2 + y^2}\mqty(x & y & \sgn(z)f(x, y)), \\
\end{align*}
obtaining\footnote{
    We are writing $\pdv{z}(\sgn(z)f(x, y)) = 0$.
    In the open set where $z \neq 0$, $\sgn(z)$ is just a constant,
    and $f$ does not depend on $z$.
    On the closed set where $z = 0$
    we can still calculate the partial derivative because we know that the function
    $N(x, y, z)$ exists and is differentiable.
    And by the proximity of points where $\sgn(z)$ is positive
    and points where $\sgn(z)$ is negative,
    the partial derivative has to be zero.
}
\begin{align*}
    2H(x, y, z) &= \divergence{N}(x, y, z) \\
        &= \frac{x^2 + y^2 - 2x^2}{(x^2 + y^2)^2} +
            \frac{x^2 + y^2 - 2x^2}{(x^2 + y^2)^2} +
            0 \\
        &= 0.
\end{align*}
Therefore, $S$ is minimal.

To complete \cref{itm:minimal-umbilical},
we need to check wheter the surface has any umbilical points,
those where both principal curvatures, $κ_1$ and $κ_2$, are zero.
We know that $κ_1 + κ_2 = 2H = 0$.
It will suffice to check whether the guassian curvature $K = κ_1κ_2$
is also zero at some point.
To that purporse, I will compute the gaussian curvature using the formula
$K = \frac{\inner{ν}{ν_θ \cross ν_t}}{\norm{ν}^4}$,
where $ν = X_θ \cross X_t$.
\begin{gather*}
    \begin{aligned}
        ν(θ, t) &= X_θ(θ, t) \cross X_t(θ, t)
            = \cosh{t}\mqty(\cos{θ} & \sin{θ} & -\sinh{t}), \\
        ν_θ(θ, t) &= \cosh{t}\mqty(-\sin{θ} & \cos{θ} & 0), \\
        ν_t(θ, t) &= \mqty(\sinh{t}\cos{θ} & \sinh{t}\sin{θ} & -\cosh^2 t - \sinh^2 t), \\
        ν_θ(θ, t) \cross ν_t(θ, t) &= -\cosh{t}\mqty(
                (\cosh^2 t + \sinh^2 t)\cos{θ} &
                (\cosh^2 t + \sinh^2 t)\sin{θ} &
                \sinh{t}
            ),
    \end{aligned} \\
    \inner{ν(θ, t)}{ν_θ(θ, t) \cross ν_t(θ, t)}
        = -\cosh^2 t\qty(\cosh^2 t + \sinh^2 t - \sinh^2 t)
        = -\cosh^4 t
\end{gather*}
and the gaussian curvature is
\begin{gather*}
    K(θ, t)
        = \frac{\inner{ν(θ, t)}{ν_θ(θ, t) \cross ν_t(θ, t)}}{\norm{ν(θ, t)}^4}
        = \frac{-\cosh^4 t}{(\cosh^2 t)^4}
        = -\frac{1}{\cosh^4 t} < 0.
\end{gather*}
Therefore, the catenoid has no umbilical points.

\newpage
%----------------------------------------------------------------------------------------

We will finish the exercise comparing the area of $S_a$ and $\D_a$.
\begin{align*}
    \area(S_a) &= 2π\qty(\frac{\sinh(2a)}{2} + a) \qq{and} \\
    \area(\D_a) &= 2π\cosh^2a.
\end{align*}
Let $f(a) = \area(S_a) - \area(\D_a)$,
\begin{align*}
    f(a) &= 2π\qty(\frac{\sinh(2a)}{2} - \cosh^2a + a)
    = 2π\qty(\frac{\sinh(2a)}{2} - \frac{\cosh(2a)}{2} + a), \\
    f'(a) &= 2π(\cosh(2a) - \sinh(2a) + 1) = 2π(e^{-2a} + 1) > 0.
\end{align*}
We can see that $f$ is strictly increasing and that $f(0) = -2π$.
There are two possibilities,
either $f$ stays negative (and $\lim_{a \to \infty} f'(a) = 0$)
or $f(L) = 0$ for a single $L \in \R$
and $f(a) > 0$ if and only if $a > L$,
as the exercise demands us to prove.
In point of fact,
the first case can be discarded because $\lim_{a \to \infty} f'(a) = 2π$.

%----------------------------------------------------------------------------------------

\end{document}
