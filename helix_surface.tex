%%%%%%%%%%%%%%%%%%%%%%%%%%%%%%%%%%%%%%%%%
% fphw Assignment
% LaTeX Template
% Version 1.0 (27/04/2019)
%
% This template originates from:
% https://www.LaTeXTemplates.com
%
% Authors:
% Class by Felipe Portales-Oliva (f.portales.oliva@gmail.com) with template 
% content and modifications by Vel (vel@LaTeXTemplates.com)
%
% Template (this file) License:
% CC BY-NC-SA 3.0 (http://creativecommons.org/licenses/by-nc-sa/3.0/)
%
%%%%%%%%%%%%%%%%%%%%%%%%%%%%%%%%%%%%%%%%%

%----------------------------------------------------------------------------------------
%    PACKAGES AND OTHER DOCUMENT CONFIGURATIONS
%----------------------------------------------------------------------------------------

\documentclass[
    12pt, % Default font size, values between 10pt-12pt are allowed
    %letterpaper, % Uncomment for US letter paper size
    %spanish, % Uncomment for Spanish
]{fphw}

% Template-specific packages
\usepackage[utf8]{inputenc} % Required for inputting international characters
\usepackage[T1]{fontenc} % Output font encoding for international characters
\usepackage{fontspec,unicode-math} % Required for using utf8 characters in math mode
\usepackage{parskip}  % To add extra space between paragraphs
% \usepackage{mathpazo} % Use the Palatino font
\usepackage{graphicx} % Required for including images
\usepackage{booktabs} % Better horizontal rules in tables
\usepackage{hyperref} % For links (both internal and external)
% \usepackage{listings} % Required for insertion of code
\usepackage{enumerate}% To modify the enumerate environment
\usepackage{cleveref} % Better \ref command -> \cref
\usepackage{import}   % This 4 packages and the command allow importing pdf
\usepackage{xifthen}  % figures generated with inkscape
\usepackage{pdfpages} % Source: https://castel.dev/post/lecture-notes-2/
\usepackage{mathtools}
\usepackage{wrapfig}
\usepackage{cancel}
\usepackage{transparent}
\newcommand{\incfig}[1]{%
    \def\svgwidth{0.95\columnwidth}
    \small
        \import{./images/}{#1.pdf_tex}
}

\setlength{\parindent}{15pt}
\setlength{\headheight}{22.66pt}

%----------------------------------------------------------------------------------------
%    ASSIGNMENT INFORMATION
%----------------------------------------------------------------------------------------

\title{Task 8 \\ Helix Surface} % Assignment title

\author{Emilio Domínguez Sánchez} % Student name

\date{December 26th, 2020} % Due date

\institute{University of Murcia \\ Faculty of Mathematics} % Institute or school name

\class{Geometría de Superficies} % Course or class name

\professor{Dr. Pascual Lucas Saorin} % Professor or teacher in charge of the assignment

%----------------------------------------------------------------------------------------
%    Definitions
%----------------------------------------------------------------------------------------

\usepackage{physics}
\newcommand{\R}{\mathbb{R}}
\newcommand{\basis}[2]{\qty{#1,\ #2}}
\DeclareMathOperator{\sgn}{sgn}
\DeclareMathOperator{\Id}{Id}
\newcommand{\inner}[2]{\left\langle #1, \; #2 \right\rangle}
\newcommand{\n}{\vectorbold{n}}

\begin{document}

\maketitle % Output the assignment title, created automatically using the information in the custom commands above

%----------------------------------------------------------------------------------------
%    ASSIGNMENT CONTENT
%----------------------------------------------------------------------------------------

\section*{Problem}

\begin{problem}
    Let $β : I \subset \R \to \R^2 \subset \R^3$ be a planar, parametrized by arc, curve
    with curvature $κ_β$ and tangent and normal vectors $T_β$ and $N_β$ respectively.
    Let $\n = (T_β \cross N_β)(s)$ be the unitary vector
    orthogonal to the plane that contains $β$.
    Then, given $φ \in [0, \frac{π}{2}]$, we define
    the \textit{helix surface $S_φ$ built over $β$}
    as the surface parametrized by
    %
    \begin{equation*}
        X(s, t) = β(s) + t(\cos{φ}N_β(s) + \sin{φ}\n).
    \end{equation*}

    \begin{enumerate}
        \item \label{stm:i} Determine the unitary field $N(s, t)$ normal to $S_φ$.

        \item \label{stm:ii} Compute the Gaussian curvature and the mean curvature
        and classify its points.

        \item \label{stm:iii} Answer whether $β$, seen as belonging to $S_φ$,
        is a line of curvature or an asymptotic curve.

        \item \label{stm:iv} What type of surface do we obtain when $φ = \frac{π}{2}$?
        And when $φ = 0$?
    \end{enumerate}
\end{problem}

%----------------------------------------------------------------------------------------

\subsection*{Answer}

    It is unclear by the statement over which domain is $X$ defined.
One option is to assume that $t$ moves over a range which makes $S_φ$ not self intersect.
Another option is to think of $S_φ$ as a surface in a manifold of higher dimension,
where it does not intersect.
That is, $X : \R^2 \to \R^5 = [\R^2 | \R^3]$,
%
\begin{equation*}
    X(s, t) = [(s, t) | β(s) + t(\cos{φ}N_β(s) + \sin{φ}\n)].
\end{equation*}
%
We can proceed without fixing a domain,
merely requiring that it is an open set in $\R^2$.

\subsubsection*{Item \ref{stm:i}}
    To compute a normal unitary field to $S_φ$,
we will normalize the cross product of $X_s$ and $X_t$.
%
\begin{align*}
    X_s(s, t) &= T_β(s) - \cos{φ}tκ_β(s)T_β(s) = (1 - \cos{φ}tκ_β(s))T_β(s) \qq{and} \\
    X_t(s, t) &= \cos{φ}N_β(s) + \sin{φ}\n,
\end{align*}
%
where we have used that $N_β'(s) = -κ_β(s)T_β(s)$.
In addition, we can also see that $\inner{X_s}{X_t} = 0$.
To perform the cross product mentally fast,
note that $T_β$, $N_β$ and $\n$ are a normal basis in $\R^3$ and hence
$T_β \cross N_β = \n$, $N_β \cross \n = T_β$ and $\n \cross T_β = N_β$\footnote{
    Actually, it is a positive oriented basis because $\n \coloneqq T_β \cross N_β$.
    Otherwise we would have the same equalitites but with a minus sign.
}.
Therefore
%
\begin{equation*}
    X_s(s, t) \cross X_t(s, t) =
    (1 - \cos{φ}tκ_β(s))(-\sin{φ}N_β(s) + \cos{φ}\n).
\end{equation*}
%
We can see that the right part, $-\sin{φ}N_β(s) + \cos{φ}\n$, is unitary,
but in order to preserve the sign of $X_s \cross X_t$,
we need to know the sign of $1 - \cos{φ}t$.

    Now it is clear that we need to study the domain of $X$,
because depending on the sign of $1 - \cos{φ}t$,
the normal is $-\sin{φ}N_β(s) + \cos{φ}\n$ or it is $\sin{φ}N_β(s) - \cos{φ}\n$.
What is more, when $φ \ne \frac{π}{2}$ and $κ_β(s) \neq 0$,
$X_s\qty(s, \frac{1}{\cos{φ}κ_β(s)}) = 0$.
This means that the value $t = \frac{1}{\cos{φ}κ_β(s)}$ is problematic because
$S_φ$ stops being a surface.
Below and above that value, we have two connected components with opposite normal vectors.
In general,
%
\begin{equation*}
    N_{S_φ} = \mp \sin{φ}N_β \pm \cos{φ}\n,
\end{equation*}
%
where the sign is given by the sign of $1 - \cos{φ}N_β(s)$.

\subsubsection*{Item \ref{stm:ii}}

The Gaussian curvature and the mean curvature are defined in terms of the
principal curvatures of the surface,
which are given by the eigen vectors of $-\dd{N_{S_φ}}$.
In this case, it will be easy to compute both from
the expression $N_{S_φ}(s, t)$ we derived.
%
\begin{align*}
        N \circ X\;(s, t) = {}&
            \mp \sin{φ}N_β(s) \pm \cos{φ}\n. \\
    \dd{N}(X(s, t))\qty(X_t(s, t)) = \qquad
        \pdv{N \circ X}{t}\,(s, t) = {}&
            0. \\
    \dd{N}(X(s, t))\qty(X_s(s, t)) = \qquad
        \pdv{N \circ X}{s}\,(s, t) = {}&
            \pm \sin{φ}κ_β(s)T_β(s). \\
\end{align*}
%
From the first equality we obtain that one eigen value is $0$,
and therefore the Gaussian curvature is also $0$,
while the second can be obtained from the last identity, because
$\dd{N}(X(s, t))\qty(X_s)$ is proportional to $X_s$
(both are an elongation of $T_β(s)$).
Divide the second result by the coefficient which acompanies $X_s$,
$1 - t\cos{φ}κ_β$,
to get the second eigen value.
%
\begin{gather*}
    \qty{κ_1(s, t), κ_2(s, t)} =
    \qty{0, \mp \frac{\sin{φ}κ_β(s)}{1 - t\cos{φ}κ_β(s)} }. \\
    K(s, t) = 0 \qquad H(s, t) = \mp \frac{\sin{φ}κ_β(s)}{2(1 - t\cos{φ}κ_β(s))}.
\end{gather*}

\subsubsection*{Item \ref{stm:iii}}

From the previous item we can also deduce that $X_t$ and $X_s$ are
eigen vectors associated with the principal curvatures.
This means that $β(s) = X(s, 0)$ is a line of curvature,
because $β'(s) = X_s(s, 0)$,
but it is not an asymptotic curve unless
%
\begin{equation*}
    -\frac{\sin{φ}κ_β(s)}{2(1 - t\cos{φ}κ_β(s))} = 0.
\end{equation*}
%
That is, when $φ = \frac{π}{2}$ or $κ_β(s) = 0$.
    
\subsubsection*{Item \ref{stm:iv}}

When $φ = \frac{π}{2}$,
%
\begin{equation*}
    X(s, t) = β(s) + t\n,
\end{equation*}
%
an extrusion of $β$ in perpendicular to the plain it is contained in.
For example, if $β$ is a fraction of a circle,
$S_{\frac{π}{2}}$ would be a fraction of a cylinder.
In other cases, it would be something that resembles a cylinder.
Let us call it a fraction of a pseudo-cylinder.

When $φ = 0$,
%
\begin{equation*}
    X(s, t) = β(s) + tN_β(s),
\end{equation*}
%
an extension of $β$ over the plane it is contained in.
$S_0$ would be contained in a plane.
That would make $S_0$ a fraction of a plane.

%----------------------------------------------------------------------------------------

\end{document}
