%%%%%%%%%%%%%%%%%%%%%%%%%%%%%%%%%%%%%%%%%
% fphw Assignment
% LaTeX Template
% Version 1.0 (27/04/2019)
%
% This template originates from:
% https://www.LaTeXTemplates.com
%
% Authors:
% Class by Felipe Portales-Oliva (f.portales.oliva@gmail.com) with template 
% content and modifications by Vel (vel@LaTeXTemplates.com)
%
% Template (this file) License:
% CC BY-NC-SA 3.0 (http://creativecommons.org/licenses/by-nc-sa/3.0/)
%
%%%%%%%%%%%%%%%%%%%%%%%%%%%%%%%%%%%%%%%%%

%----------------------------------------------------------------------------------------
%    PACKAGES AND OTHER DOCUMENT CONFIGURATIONS
%----------------------------------------------------------------------------------------

\documentclass[
    12pt, % Default font size, values between 10pt-12pt are allowed
    %letterpaper, % Uncomment for US letter paper size
    %spanish, % Uncomment for Spanish
]{fphw}

% Template-specific packages
\usepackage[utf8]{inputenc} % Required for inputting international characters
\usepackage[T1]{fontenc} % Output font encoding for international characters
\usepackage{fontspec,unicode-math} % Required for using utf8 characters in math mode
\usepackage{parskip}  % To add extra space between paragraphs
% \usepackage{mathpazo} % Use the Palatino font
\usepackage{graphicx} % Required for including images
\usepackage{booktabs} % Better horizontal rules in tables
\usepackage{hyperref} % For links (both internal and external)
% \usepackage{listings} % Required for insertion of code
\usepackage{enumerate}% To modify the enumerate environment
\usepackage{cleveref} % Better \ref command -> \cref
\usepackage{import}   % This 4 packages and the command allow importing pdf
\usepackage{xifthen}  % figures generated with inkscape
\usepackage{pdfpages} % Source: https://castel.dev/post/lecture-notes-2/
\usepackage{mathtools}
\usepackage{wrapfig}
\usepackage{cancel}
\usepackage{transparent}
\newcommand{\incfig}[1]{%
    \def\svgwidth{0.95\columnwidth}
    \small
        \import{./images/}{#1.pdf_tex}
}

\setlength{\parindent}{15pt}
\setlength{\headheight}{22.66pt}

%----------------------------------------------------------------------------------------
%    ASSIGNMENT INFORMATION
%----------------------------------------------------------------------------------------

\title{Task 1 \\ Parallel Vector Fields} % Assignment title

\author{Emilio Domínguez Sánchez} % Student name

\date{February 21st, 2021} % Due date

\institute{University of Murcia \\ Faculty of Mathematics} % Institute or school name

\class{Geometría Global de Superficies} % Course or class name

\professor{Dr. Luis J. Alías Linares} % Professor or teacher in charge of the assignment

%----------------------------------------------------------------------------------------
%    Definitions
%----------------------------------------------------------------------------------------

\usepackage{physics}
\newcommand{\R}{\mathbb{R}}
\newcommand{\inner}[2]{\left\langle #1, \; #2 \right\rangle}
\newcommand{\vfield}{\mathfrak{X}}

\DeclareDocumentCommand\covariantderivative{ s o m g d() }
{ % Covariant derivative
	% s: star for \flatfrac flat covariant derivative
	% o: optional n for nth covariant derivative
	% m: mandatory (x in Df/dx)
	% g: optional (f in Df/dx)
	% d: long-form D/dx(...)
	\IfBooleanTF{#1}
	{\let\fractype\flatfrac}
	{\let\fractype\frac}
	\IfNoValueTF{#4}
	{
		\IfNoValueTF{#5}
		{\fractype{D \IfNoValueTF{#2}{}{^{#2}}}{\diffd #3\IfNoValueTF{#2}{}{^{#2}}}}
		{\fractype{D \IfNoValueTF{#2}{}{^{#2}}}{\diffd #3\IfNoValueTF{#2}{}{^{#2}}} \argopen(#5\argclose)}
	}
	{\fractype{D \IfNoValueTF{#2}{}{^{#2}} #3}{\diffd #4\IfNoValueTF{#2}{}{^{#2}}}}
}
\DeclareDocumentCommand\cdv{}{\covariantderivative} % Shorthand for \covariantderivative

\begin{document}

\maketitle % Output the assignment title, created automatically using the information in the custom commands above

%----------------------------------------------------------------------------------------
%    ASSIGNMENT CONTENT
%----------------------------------------------------------------------------------------

\section*{Problem}

\begin{problem}
    Let $α : I \to S \subset \R^3$ be a parametrized curve on
    a regular surface oriented by a Gauss application $N$
    and let $V \in \vfield(α)$ be a differentiable tangent vector field along $α$.
    As it is usual,
    let $N(t) = N(α(t))$ denote the Gauss application with respect to the interval $I$
    and let $JV \in \vfield(α)$ denote the rotation of angle $\frac{π}{2}$ of $V$.
    That is,
    %
    \begin{equation*}
        JV(t) = N(t) \cross V(t) \in T_{α(t)}S, \qq{for all $t \in I$.}
    \end{equation*}

    \begin{enumerate}
        \item Prove that
        %
        \begin{equation*}
            \cdv{(JV)}{t} = J\qty(\cdv{V}{t}).
        \end{equation*}
        %
        In other words, that the covariant derivative commutes with $J$,
        the rotation of angle $\frac{π}{2}$.
        %
        \begin{equation*}
            \cdv{t} \circ J = J \circ \cdv{t}.
        \end{equation*}

        \item Conclude that $V \in \vfield(α)$ is parallel along $α$
        if and only if so is $JV$.

        \item Suppose that $V \neq 0$ is parallel along $α$.
        Prove that any other field $W \in \vfield(α)$ is parallel along $α$
        if and only if there exist constants $a,b \in \R$ such that
        %
        \begin{equation*}
            W(t) = aV(t) + bJV(t).
        \end{equation*}
    \end{enumerate}
\end{problem}

%----------------------------------------------------------------------------------------

\subsection*{Answer}

    Write $JV = N \cross V$. Then the derivative can be computed as
%
\begin{equation*}
    \dv{(N \cross V)}{t} = \dv{N}{t} \cross V + N \cross \dv{V}{t}.
\end{equation*}

Let us examine the first cross product.
$N$ is unitary, which means that $\dv{N}{t}$ is tangent to $S$.
Because $V$ is also tangent to $S$,
$\dv{N}{t} \cross V$ is normal to $S$.
Now let us look at the second cross product,
which is normal to $N$ and therefore tangential to $S$.
%
\begin{gather*}
    % \dv{V}{t} = \cdv{V}{t} + \qty(\dv{V}{t})^{\perp}. \\
    N \cross \dv{V}{t} =
    N \cross \qty(\cdv{V}{t} + \qty(\dv{V}{t})^{\perp}) =
    N \cross \cdv{V}{t}.
\end{gather*}
%
Writing everything together we get
%
\begin{equation*}
    \cdv{(JV)}{t} =
    \cdv{(N \cross V)}{t} =
    \qty(\dv{(N \cross V)}{t})^{\top} =
    N \cross \dv{V}{t} =
    N \cross \cdv{V}{t} =
    J\qty(\cdv{V}{t}).
\end{equation*}

    As a small corollary, we can conclude that $V$ is parallel if and only if so is $JV$,
due to the fact that $J$ is an isomorphism ($Jv = 0 \iff v = 0$).
%
\begin{equation*}
    \cdv{(JV)}{t} = J\qty(\cdv{V}{t}) = 0 \iff \cdv{V}{t} = 0.
\end{equation*}

    If we now take a parallel vector field $V \in \vfield(α)$
such that $V \neq 0$,\footnote{
    Because the norm of a parallel vector field is constant,
    it is the same to ask that $V \neq 0$ or that $V(t) \neq 0$ for all $t \in I$.
}
$V(t)$ and $JV(t)$ form an (orthogonal) basis of $T_{α(t)}S$.
Thus, any other differentiable vector field $W \in \vfield(α)$ can be written as
%
\begin{equation*}
    W = aV + bJV
\end{equation*}
%
for some \emph{unique} functions $a,b : I \to \R$.
More precisely, we know that
$a(t) = \frac{\inner{W(t)}{V(t)}}{\norm{V(t)}^2}$ and
$b(t) = \frac{\inner{W(t)}{JV(t)}}{\norm{JV(t)}^2}$,
which means that $a$ and $b$ are differentiable.
This allows us to write the covariant derivative of $W$ as
%
\begin{equation*}
    \cdv{W}{t} =
    \dv{a}{t}V + a\cdv{V}{t} + \dv{b}{t}JV + b\cdv{(JV)}{t} =
    \dv{a}{t}V + \dv{b}{t}JV.
\end{equation*}
%
$W$ is parallel if and only if its covariant derivative is everywhere $0$.
However, because $V(t)$ and $JV(t)$ form a basis,
$\cdv{W}{t}\qty(t) = \dv{a}{t}\qty(t)V(t) + \dv{b}{t}\qty(t)JV(t) = 0$
if and only if $\dv{a}{t}(t) = \dv{b}{t}(t) = 0$.
That is, $W$ is parallel if and only if $\dv{a}{t} = \dv{b}{t} = 0$
if and only if $a$ and $b$ are constants.

%----------------------------------------------------------------------------------------

\end{document}
