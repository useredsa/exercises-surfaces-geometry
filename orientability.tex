%%%%%%%%%%%%%%%%%%%%%%%%%%%%%%%%%%%%%%%%%
% fphw Assignment
% LaTeX Template
% Version 1.0 (27/04/2019)
%
% This template originates from:
% https://www.LaTeXTemplates.com
%
% Authors:
% Class by Felipe Portales-Oliva (f.portales.oliva@gmail.com) with template 
% content and modifications by Vel (vel@LaTeXTemplates.com)
%
% Template (this file) License:
% CC BY-NC-SA 3.0 (http://creativecommons.org/licenses/by-nc-sa/3.0/)
%
%%%%%%%%%%%%%%%%%%%%%%%%%%%%%%%%%%%%%%%%%

%----------------------------------------------------------------------------------------
%    PACKAGES AND OTHER DOCUMENT CONFIGURATIONS
%----------------------------------------------------------------------------------------

\documentclass[
    12pt, % Default font size, values between 10pt-12pt are allowed
    %letterpaper, % Uncomment for US letter paper size
    %spanish, % Uncomment for Spanish
]{fphw}

% Template-specific packages
\usepackage[utf8]{inputenc} % Required for inputting international characters
\usepackage[T1]{fontenc} % Output font encoding for international characters
\usepackage{fontspec,unicode-math} % Required for using utf8 characters in math mode
\usepackage{parskip}  % To add extra space between paragraphs
% \usepackage{mathpazo} % Use the Palatino font
\usepackage{graphicx} % Required for including images
\usepackage{booktabs} % Better horizontal rules in tables
\usepackage{hyperref} % For links (both internal and external)
% \usepackage{listings} % Required for insertion of code
\usepackage{enumerate}% To modify the enumerate environment
\usepackage{cleveref} % Better \ref command -> \cref
\usepackage{import}   % This 4 packages and the command allow importing pdf
\usepackage{xifthen}  % figures generated with inkscape
\usepackage{pdfpages} % Source: https://castel.dev/post/lecture-notes-2/
\usepackage{mathtools}
\usepackage{wrapfig}
\usepackage{cancel}
\usepackage{transparent}
\newcommand{\incfig}[1]{%
    \def\svgwidth{0.95\columnwidth}
    \small
        \import{./images/}{#1.pdf_tex}
}

\setlength{\parindent}{15pt}
\setlength{\headheight}{22.66pt}

%----------------------------------------------------------------------------------------
%    ASSIGNMENT INFORMATION
%----------------------------------------------------------------------------------------

\title{Task 7 \\ Orientability} % Assignment title

\author{Emilio Domínguez Sánchez} % Student name

\date{November 21th, 2020} % Due date

\institute{University of Murcia \\ Faculty of Mathematics} % Institute or school name

\class{Geometría de Superficies} % Course or class name

\professor{Dr. Pascual Lucas Saorin} % Professor or teacher in charge of the assignment

%----------------------------------------------------------------------------------------
%    Definitions
%----------------------------------------------------------------------------------------

\usepackage{physics}
\newcommand{\R}{\mathbb{R}}
\newcommand{\basis}[2]{\qty{#1,\ #2}}
\DeclareMathOperator{\sgn}{sgn}
\DeclareMathOperator{\Id}{Id}

\begin{document}

\maketitle % Output the assignment title, created automatically using the information in the custom commands above

%----------------------------------------------------------------------------------------
%    ASSIGNMENT CONTENT
%----------------------------------------------------------------------------------------

\section*{Problem}

\begin{problem}
    \begin{enumerate}
        \item \label{stm:i} Prove that if $φ : S_1 \to S_2$ is
        a differentiable application between two surfaces that is also
        a local diffeomorphism in every point $p \in S_1$ and
        $S_2$ is orientable then
        $S_1$ is orientable.

        \item \label{stm:ii} Prove that if $φ : S_1 \to S_2$ is a diffeomorphism.
        then $S_1$ is orientable if and only if $S_2$ is orientable.
        (Therefore, orientability is a quality invariant under a diffeomorphism.)

        \item \label{stm:iii} Prove that if $φ : S_1 \to S_2$ is a diffeomorphism between
        two oriented surfaces,
        $φ$ may induce a different orientation over $S_2$ than the initial.
    \end{enumerate}
\end{problem}

%----------------------------------------------------------------------------------------

\subsection*{Answer}

    \Cref{stm:ii} is a consequence of \cref{stm:i}.
A diffeomorphism and its inverse are local diffeomorphisms.

    We will use the definition that
a surface $S$ is orientable if there is an atlas $A_S$ such that the
differential of the change of coordinates
between any pair of parametrizations that intersect
has a positive determinant.

    That means that the bases of the tangent plane $T_p(S)$
over a point $p \in S$ given
by any two intersecting parametrizations $X,Y \in A_S$
determine the same orientation of $T_p$,
because the change of coordinates
%
\begin{equation*}
    \mqty(X_{\bar{u}} \\ X_{\bar{v}}) =
    { \displaystyle
    \mqty(
        \pdv{\bar{u}}{u} & \pdv{\bar{u}}{v} \\[6pt]
        \pdv{\bar{v}}{u} & \pdv{\bar{v}}{v}
    )
    }
    \mqty(X_u \\ X_v)
\end{equation*}
%
(at the point $p$)
is the jacobian in $p$ of the change of coordinates between $X$ and $Y$.

    We start the exercise fixing an orientation for $S_2$ by picking
any fixed atlas $A_{S_2}$.
The orientation defined on the tangent planes of $S_2$
by any parametrization $X \in A_{S_2}$ will be called \emph{positive}.

    We can use $φ$ to select an atlas on $S_1$ that is oriented as well.
For every point $p \in S_1$,
$\dd{φ}(p)$ is a biyection between $T_p(S_1)$ and $T_{φ(p)}(S_2)$.
Hence, we will call a parametrization $X : U \to X(U)$
\emph{positively oriented} if the basis
$\basis{\dd{φ}(X(q))(X_u)}{\dd{φ}(X(q))(X_v)}$
is positively oriented in $T_{φ(q)}(S_2)$ for any $q \in U$.

    Given two intersecting positively oriented parametrizations $X$ and $Y$
of $p \in S$,
the jacobian of the change of coordinates between them must be positive,
because if $M$ is the positive operator that acts as the change of coordinates between
$\basis{\dd{φ}(X_u)}{\dd{φ}(X_v)}$ and $\basis{\dd{φ}(Y_u)}{\dd{φ}(U_v)}$,
the change of coordinates from
$\basis{X_u}{X_v}$ and $\basis{Y_u}{Y_v}$,
%
\begin{equation*}
    \mqty(Y_u & Y_v) =
    \dd{φ}^{-1} \circ M \circ \dd{φ}
    \qty(\mqty(X_u & X_v)), \\
\end{equation*}
%
which is equal to the jacobian, is positive:
%
\begin{equation*}
    \sgn(\dd{φ}^{-1} \circ M \circ \dd{φ}) =
    \sgn(\dd{φ}^{-1})\sgn(M)\sgn(\dd{φ}) =
    \sgn{M}.
\end{equation*}

    It is left to see that there is always
a positively oriented parametrization for any point $p \in S_1$.
Take $X : U \to X(U)$, a parametrization of $p$.
We can assume that $\basis{\dd{φ}(p)(Y_u(p))}{\dd{φ}(p)(Y_v(p))}$
is postively oriented in $T_{φ(p)}(S_2)$,
because the parametrization $Y(u, v) = X(v, u)$ satisfies
%
\begin{equation*}
    \sgn(\basis{\dd{φ}(Y_u)}{\dd{φ}(Y_v)}) =
    \sgn(\basis{\dd{φ}(X_v)}{\dd{φ}(X_u)}) =
    -\sgn(\basis{\dd{φ}(X_u)}{\dd{φ}(X_v)}).
\end{equation*}
%
Now note that
%
\begin{equation*}
    \basis{\dd{φ}(Y_u)}{\dd{φ}(Y_v)} =
    \dd{φ}\circ\dd{X}\qty(\basis{\mqty(1 \\ 0)}{\mqty(0 \\ 1)}).
\end{equation*}
%
It is clear from the equality above that the basis is positive
when the determinant of $\dd{φ}\circ\dd{X}$ is.
But because it is positive in $p$ and $\dd{φ}\circ\dd{X}$ is always of maximum rank,
it is positive in the connected component of $p$ in $U$.

    Hence, an oriented atlas for $S_1$ is given by taken
a positively oriented parametrization $X_p$ defined in a connected open set
for every point $p \in S_1$.

    An example for \cref{stm:iii} needs of two orientations for the same surface $S_2$.
And for simplicity, we will take $S_1 = S_2$ and the identity map.
The easiest surface must be $\R^2 \subset \R^3$.
The atlas $\qty{i : \R^2 \to \R^3}$ has a different orientation than
$\qty{r(u, v) = \mqty(v & u & 0) : \R^2 \to \R^3}$.
This can be justified writing the bases
%
\begin{align*}
    \basis{i_u(0)}{i_v(0)} &= \basis{e_1}{e_2} \qq{and} \\
    \basis{r_u(0)}{r_v(0)} &= \basis{e_2}{e_1},
\end{align*}
%
which have different orientations.

    The identity map $\Id : (\R^2, \{i\}) \to (\R^2 \{r\})$
is a diffeomorphism between $S_1 = \R^2$ to $S_2 = \R^2$ that induces a different
orientation in $S_2$,
as we will prove.
Fix $p \in S_2$, then, according to $\Id$,
a basis $\basis{X_u}{X_v}$ of $T_p(S_2)$ is positively oriented if
$\basis{\dd{i^{-1}}(X_u)}{\dd{i^{-1}}(X_v)} = \basis{X_u}{X_v}$ is positive in $S_1$,
where we are identifying $T_p(S_1)$ with $T_p(S_2)$.
But the bases that are positive in $S_1$ are the bases that are negative in $S_2$,
according to its original orientability.

%----------------------------------------------------------------------------------------

\end{document}
