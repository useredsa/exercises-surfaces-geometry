%%%%%%%%%%%%%%%%%%%%%%%%%%%%%%%%%%%%%%%%%
% fphw Assignment
% LaTeX Template
% Version 1.0 (27/04/2019)
%
% This template originates from:
% https://www.LaTeXTemplates.com
%
% Authors:
% Class by Felipe Portales-Oliva (f.portales.oliva@gmail.com) with template 
% content and modifications by Vel (vel@LaTeXTemplates.com)
%
% Template (this file) License:
% CC BY-NC-SA 3.0 (http://creativecommons.org/licenses/by-nc-sa/3.0/)
%
%%%%%%%%%%%%%%%%%%%%%%%%%%%%%%%%%%%%%%%%%

%----------------------------------------------------------------------------------------
%    PACKAGES AND OTHER DOCUMENT CONFIGURATIONS
%----------------------------------------------------------------------------------------

\documentclass[
    12pt, % Default font size, values between 10pt-12pt are allowed
    %letterpaper, % Uncomment for US letter paper size
    %spanish, % Uncomment for Spanish
]{fphw}

% Template-specific packages
\usepackage[utf8]{inputenc} % Required for inputting international characters
\usepackage[T1]{fontenc} % Output font encoding for international characters
\usepackage{fontspec,unicode-math} % Required for using utf8 characters in math mode
\usepackage{parskip}  % To add extra space between paragraphs
% \usepackage{mathpazo} % Use the Palatino font
\usepackage{graphicx} % Required for including images
\usepackage{booktabs} % Better horizontal rules in tables
\usepackage{hyperref} % For links (both internal and external)
% \usepackage{listings} % Required for insertion of code
\usepackage{enumerate}% To modify the enumerate environment
\usepackage{cleveref} % Better \ref command -> \cref
\usepackage{import}   % This 4 packages and the command allow importing pdf
\usepackage{xifthen}  % figures generated with inkscape
\usepackage{pdfpages} % Source: https://castel.dev/post/lecture-notes-2/
\usepackage{mathtools}
\usepackage{wrapfig}
\usepackage{cancel}
\usepackage{transparent}
\newcommand{\incfig}[1]{%
    \def\svgwidth{0.95\columnwidth}
    \small
        \import{./images/}{#1.pdf_tex}
}

\setlength{\parindent}{15pt}
\setlength{\headheight}{22.66pt}

%----------------------------------------------------------------------------------------
%    ASSIGNMENT INFORMATION
%----------------------------------------------------------------------------------------

\title{Task 6 \\ Surface Gradient} % Assignment title

\author{Emilio Domínguez Sánchez} % Student name

\date{November 14th, 2020} % Due date

\institute{University of Murcia \\ Faculty of Mathematics} % Institute or school name

\class{Geometría de Superficies} % Course or class name

\professor{Dr. Pascual Lucas Saorin} % Professor or teacher in charge of the assignment

%----------------------------------------------------------------------------------------
%    Definitions
%----------------------------------------------------------------------------------------

\usepackage{physics}
\newcommand{\R}{\mathbb{R}}
\newcommand{\inner}[2]{\left\langle #1, \; #2 \right\rangle}
\newcommand{\tf}{\tilde{f}}
\DeclareMathOperator{\Ima}{Im}
\DeclareMathOperator{\proj}{proj}

\begin{document}

\maketitle % Output the assignment title, created automatically using the information in the custom commands above

%----------------------------------------------------------------------------------------
%    ASSIGNMENT CONTENT
%----------------------------------------------------------------------------------------

\section*{Problem}

\begin{problem}

    The \emph{gradient} of a differentiable function $f : S \to \R$ is the function
$\grad{f} : S \to \R^3$ that assigns each point $p \in S$ the vector
$\grad{f}(p) \in T_p \subset \R^3$ such that

\begin{equation*}
    \braket{\grad{f}(p)}{v} = \dd{f}_p(v)
\end{equation*}

\noindent
for each $v \in T_pS$.
The gradient is a good example of a field of tangent vectors over $S$.

    Prove that

\begin{enumerate}
    \item If $E$, $F$, and $G$ are the coefficients of the first fundamental form
    for a parametrization $X : U \subset \R^2 \to S$, then

    \begin{equation*}
        \grad{f} = \frac{\tf_uG - \tf_vF}{EG-F^2}X_u + \frac{\tf_vE - \tf_uF}{EG-F^2}X_v
    \end{equation*}

    inside $X(U)$.
    Where $\tf = f \circ X$ is the local representative of $f$ by $X$.

    And in particular, if $S = \R^2$ then the definition matches the usual definition
    of gradient for functions defined over $\R^2$.

    \item The gradient for a surface points in the direction (in the tangent space)
    of greatest slope.

    \begin{equation*}
        \max\qty{\frac{\dd{f}_p(v)}{\norm{v}}}_{v \in T_pS} =
        \max\qty{\dd{f}_p\qty(\frac{v}{\norm{v}})}_{v \in T_pS} =
        \dd{f}_p\qty(\frac{\grad{f}(p)}{\norm{\grad{f}(p)}}).
    \end{equation*}

    \item If $\grad{f} ≠ 0$ for every point of a contour line $C = f^{-1}(a)$,
    then $C$ is a regular curve over $S$ and $\grad{f}$ is orthogonal to $C$ in $C$.

    \begin{equation*}
        0 \notin \grad{f}(C) \implies
        \begin{dcases}
            \text{$C$ is a curve} \\
            \grad{f} \perp C.
        \end{dcases}
    \end{equation*}
\end{enumerate}

\end{problem}

%----------------------------------------------------------------------------------------

\subsection*{Answer}

    \textit{Riesz Theorem} states that for any linear functional $F$
over a finite dimensional\footnotemark Hilbert space (like $T_pS$),
there exists a vector $v$ such that $F = \braket{v}{\cdot}$.
This theorem justifies the definition of the differential.
That the gradient points in the direction of greatest slope is a consequence of
\textit{Cauchy-Schwarz} inequality.
%
\begin{equation*}
    \abs{\braket{\grad{f}(p)}{\frac{v}{\norm{v}}}} ≤ \norm{\grad{f}(p)}.
\end{equation*}
%
Which is an equality when
$\braket{\grad{f}(p)}{\frac{\grad{f}(p)}{\norm{\grad{f}(p)}}} = \norm{\grad{f}(p)} ≥ 0$.
Therefore,
%
\begin{equation*}
    \max\qty{\frac{\dd{f}_p(v)}{\norm{v}}}_{v \in T_pS} =
    \frac{\dd{f}_p(\grad{f}(p))}{\norm{\grad{f}(p)}}.
\end{equation*}

\footnotetext{Actually, the theorem can also be proved
for infinite dimensional Hilbert spaces
if the linear functional is continuous.}

    Next we will find the expression of $f$ in the basis $\qty{X_u, X_v}$.
%
\begin{equation*}
    \grad{f} = c_uX_u + c_vX_v.
\end{equation*}
%
Given that we have an expression for the dot product of $\grad{f}$,
our mind could be asking us to write
$c_u = \braket{\grad{f}}{X_u}$ and $c_v = \braket{\grad{f}}{X_v}$.
And although that is not true because $X_u$ and $X_v$ are not necesarilly ortonormal,
we can still deduce the values of $c_u$ and $c_v$ using those products,
because
%
\begin{alignat*}{2}
    \braket{\grad{f}}{X_u} &=
    \braket{c_uX_u + c_vX_v}{X_u} &=
    c_u\braket{X_u}{X_u} + c_v\braket{X_v}{X_u} &=
    Ec_u + Fc_v \qq{and} \\
    \braket{\grad{f}}{X_v} &=
    \braket{c_uX_u + c_vX_v}{X_v} &=
    c_u\braket{X_u}{X_v} + c_v\braket{X_v}{X_v} &=
    Fc_u + Gc_v.
\end{alignat*}
%
Which, using the differentail equal
%
\begin{equation*}
    \begin{aligned}
        \braket{\grad{f}}{X_u} &=
        \dd{f}(X_u) &= \\
        \braket{\grad{f}}{X_v} &=
        \dd{f}(X_v) &=
    \end{aligned}
    \qq{(by definition of the differential)\footnote{}}
    \begin{aligned}
        = &\tf_u &\qq{and} \\
        = &\tf_v.
    \end{aligned}
\end{equation*}

\footnotetext{Or, reproducing the definitions, if so you prefer:
$\dd{f}(X_u) = \dd{f} \circ \dd{X}(1, 0) = \dd{\tf}(1, 0) = \tf_u$.}

     Matching both expressions we get the system
%
\begin{align*}
    \begin{dcases}
        c_uE + c_vF = \tf_u \\
        c_uF + c_vG = \tf_v
    \end{dcases}
    \implies
    \begin{dcases}
        c_u(EG - F^2) = \tf_uG - \tf_vF \\
        c_v(F^2 - GE) = \tf_uF - \tf_vE \\
    \end{dcases},
    & \qq{which,} \\
    \qq{if $EG-F^2 ≠ 0$, implies that} &
    \begin{dcases}
        c_u = \frac{\tf_uG - \tf_vF}{EG-F^2}X_u \\
        c_v = \frac{\tf_vE - \tf_uF}{EG-F^2}X_v &.
    \end{dcases}
\end{align*}
%
But $EG-F^2$ cannot be zero, ultimately because $\qty{X_u, X_v}$ is a basis of $T_pS$
which in turn is because $\dd{X}$ is inyective.

    When the basis is taken to be orthogonal the expression is greatly simplified.
%
\begin{equation*}
    \grad{f} =
    \frac{\tf_uG - \tf_vF}{EG-F^2}X_u + \frac{\tf_vE - \tf_uF}{EG-F^2}X_v =
    (F = 0) =
    \tf_u\frac{X_u}{E} + \tf_v\frac{X_v}{G}.
\end{equation*}
%
And in particular, when $S = \R^2$ and $X$ is taken to be the identity so that
$X_u = \mqty(1 & 0)$ and $X_v = \mqty(0 & 1)$,
$E = G = 1$, and the equation can be written as
%
\begin{multline*}
    \grad{f} =
    \tf_u\mqty(1 & 0) + \tf_v\mqty(0 & 1) =
    \mqty(\tf_u & \tf_v) = \\
%
    \begin{multlined}
\text{\Big(Where $\tf_u = \dd(f\circ X)\mqty(1 & 0) = \dd{f}\mqty(1 & 0) = \pdv{f}{u}$} \\
\text{using the standard definition of differential.\Big)}
    \end{multlined} \\
%
    \mqty(\pdv{f}{u} & \pdv{f}{v}) =
    \grad{f},
\end{multline*}
%
where the last symbol is the traditional interpretation of the gradient.

    Lastly, if $\grad{f} ≠ 0$ for every point of a contour line $C = f^{-1}(a)$,
$C$ should be a regular curve.
There is a similar result over $\R^2$ that proves it using the implicit function theorem.
But, as usual when developing the theory about surfaces,
we do not have it defined or proved for surfaces.
Luckily, the solution is, as usual, bringing the problem to $\R^2$
making use of $X$.

    Fix $p \in C$. We would like to start by proving that
$\grad{f}(p) ≠ 0 \iff \grad{\tf}(X^{-1}(p)) ≠ 0$.
Such an intuitive statement would be hard to prove playing with the
(clearly not-)inspiring expression we have derived for $\grad{f}(p)$.
Instead, it is easier to prove that $\grad{f}(p) = 0 \iff \grad{\tf}(X^{-1}(p)) = 0$.
%
\begin{multline*}
    \grad{f}(p) = 0 \iff
    \dd{f}(p) = 0 \iff
    \qty(\dd{X} biyective) \iff \\
    \dd{f}(p)\circ\dd{X}(X^{-1}(p)) =
        \dd(f\circ X)(X^{-1}(p)) = \dd{\tf}(X^{-1}(p)) = 0 \iff \\
    \grad{\tf}(X^{-1}(p)) = 0.
\end{multline*}

    Now we will move the problem to $U$.
Define $\tilde{C} = X^{-1}(C\cap \Ima{X})$.
From here onwards, we write $q = X^{-1}(p)$ for simplicity.
Clearly, $q$ is in $\tilde{C}$.
Because $\grad{\tf}(q) ≠ 0$, either $\pdv{\tf}{u}\,(q) ≠ 0$ or $\pdv{\tf}{v}\,(q) ≠ 0$.
Without loss of generality, assume it is $\pdv{\tf}{v}\,(q)$.
Then there exists a neighborhood $N$ of $q$ ($\in \R^2$) and
a differentiable function $v : (I := \proj_{\R}(N)) \to N$ such that
$\qty{\mqty(u & v(u))}_{u \in I} = \tilde{C} \cap N$.
That means that locally, $\tilde{C} = \tilde{α}(I)$ where $\tilde{α}(t) = \mqty(t & v(t))$.
And we can return our conclusion to $C$ writing
%
\begin{equation*}
    C \cap \qty(\Ima X \cap X(N)) =
    \qty(C \cap \Ima X) \cap X(N) =
    X(\tilde{C} \cap N) =
    X(\tilde{α}(I)).
\end{equation*}
%
Note that $X(N)$ is a neighborhood of $p$ because $X$ is an homeomorphism (thus it is open).
Hence, we have proved that there exists a neighborhood, $\Ima X \cap X(N)$, of $p$
on which $C$ is $X(α(I))$,
for $N$, $I$ and $α$ all dependent on $X$ and $p$.
And in order to complete the argument,
we emphasize that $\tilde{α} = \qty[t \mapsto \mqty(t & v(t))]$ is clearly
differentiable, regular\footnotemark and inyective,
and so would be $α(t) := X(\tilde{α}(t))$,
completing the argument that $C$ is a regular curve in $S$.

\footnotetext{Indeed, $\dv{\tilde{α}}{t} = \mqty(1 & \dv{v}{t}) ≠ 0$.
Also, the inyectivity is key to prove that $\tilde{α}$ is an homeomorphism.}

    In the previous conditions, $f(α(t)) = a$ is a constant function from $\R$ to $\R$.
%
\begin{equation*}
    0 = \dd{(f \circ α)}(t) = \dd{f}(α(t))(\dd{α}(t)) = \braket{\grad{f}(α(t))}{\dd{α}(t)}
    \implies
    \grad{f}(p) \perp T_pC.
\end{equation*}
%
Where we have used the chain rule for functions defined between manifolds,
the definition of gradient and
the fact that the image of $\dd{α}(t)$ generates $T_{α(t)}C$.

%----------------------------------------------------------------------------------------

\end{document}
